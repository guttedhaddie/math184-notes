\graphicspath{{6indo-islamic/pics/},{6indo-islamic/asy/}}

\section{Indian and Islamic Mathematics}

\subsection{India, Hindu--Arabic Numerals \& Zero}

\begin{minipage}[t]{0.65\linewidth}\vspace{-6pt}
	The Indian/South Asian subcontinent is bordered to the north by the Himalayan mountains and to the east by dense jungle. Its primary frontier has historically been the fertile Indus valley in the west, now the central corridor of Pakistan, where recorded civilization dates to at least 2500\BC. During the first millennium \!\BC{}, Hinduism developed as an amalgamation of previous practices and beliefs; Buddhism and Jainism began to spread in the later part of this period, primarily in the Ganges valley further east.
	\smallbreak

	Alexander the Great conquered as far east as the Indus in 326\BC, bringing Greek, Babylonian and Egyptian knowledge in his wake. The Greek overlords he left behind were rapidly overthrown and \phantom{the subcontinent became largely unified under the Maurya}
\end{minipage}
\hfill
\begin{minipage}[t]{0.35\linewidth}\vspace{-7pt}
	\flushright
	\includegraphics[scale=0.2]{Ancient_India_Map}
\end{minipage}\par
\vspace{-12pt}
the subcontinent became largely unified under the Mauryan Empire for the next 150 years.  After this came 1000 years of shifting control with several invasions from the west by the nearby Persians. The expanding Islamic caliphate conquered the Indus around \AD 1000, and the greater part of India was unified under the Islamic Mughal Empire by the 1500s. After its decline and fragmentation, the British conquered India in 1857.
\smallbreak

The modern political situation reflects this complicated history. India gained its independence in 1947 after World War II, and was partitioned according to religion: the greater Indus valley and the lower Ganges/Brahmaputra comprise the Islamic states of Pakistan and Bangladesh, with the majority of the landmass becoming the nominally secular but majority Hindu \emph{country} of India. The upper Indus valley (Kashmir) remains contested and has been the site of several military conflicts between India, Pakistan and China.
\smallbreak

Ancient India is important not just for its separate contributions, but as a technological and cultural crossroads: it is estimated that it accounted for 25--30\% of the world's economy during the 1\st{} millenium \!\AD{}! India's location made it well suited to absorb and synthesise ideas and technologies from both east and west; while some trade and knowledge passed north of the Himalayas directly between China and the Middle East/Europe, far more percolated slowly through India, being improved and given back in turn.


% \subsubsection*{Kharosthi numerals}
% 
% Derived from Aramaic script dating 4th--2nd C BC.
% 
% Special symbols for 10, 20.
% 
% Build to 100 additively.
% 
% 100s and larger used special symbols for powers of 10.
% 
% %Written right to left as common in West Asia.\\

\boldinline{Brahmi Numerals \& Numerical Naming}

Our primary focus is on possibly the most important practical mathematical development in history: the decimal positional system of enumeration, complete with fully-functional zero. The Brahmi numerals, one of the earliest antecedents of modern numerals, first appeared around the 3\rd{} century \BC{}.
\begin{center}
	\begin{tabular}{cccccccccc}
		1&2&3&4&5&6&7&8&9&10\\
		\IndiaBone&\IndiaBtwo&\IndiaBthree&\IndiaBfour&\IndiaBfive&\IndiaBsix&\IndiaBseven&\IndiaBeight&\IndiaBnine&\IndiaBten
	\end{tabular}
\end{center}
The example dates from around 100\BC{} and was used in Mumbai/Bombay. Additional symbols denoted multiples of 10, 100, 1000, 10000, etc. As with Chinese characters, the system was partly positional (800 would be written by prefixing the symbol for 100 by that for 8) and there was no symbol or placeholder for zero.
\goodbreak

Symbols are only part of the story. The modern approach to naming numbers and constructing large numbers can also be linked to the same period. The table below gives old Sanskrit names.
\begin{center}
	\begin{tabular}{ccccccccc}
		1&2&3&4&5&6&7&8&9\\
		eka&dvi&tri&catur&pancha&sat&sapta&asta&nava\\[0.3cm]
		10&20&30&40&50&60&70&80&90\\
		dasa&vimsati&trimsati&catvarimsat&panchasat&sasti&saptati&asiti&navati\\[0.3cm]
		100&1000&10000&100000&1000000&$10^7$&$10^8$&$10^9$&$10^{10}$\\
		sata&sahasra&ayuta&niyuta&prayuta&arbuda&nyarbuda&samudra&madhya
	\end{tabular}
\end{center}
You should recognize similarities with some of the numbers in European languages, many of which have Indian roots. The construction of larger numbers is also be familiar: for example \emph{tri sahasra sat sata panchasat nava} is precisely how we read 3659.
\smallbreak
There were also plenty of differences with modern verbiage. Sanskrit had distinct words for powers of 10 up to (at least!) $10^{62}$. They also had a version of pre-subtraction: for example \emph{ekanna-niyuta} meant `one less than 100000,' or 99999.


\boldinline{Gwalior Numerals}

During the first few centuries \AD{}, a fully positional decimal place system came into being. The earliest evidence comes from a manuscript found in \href{http://www.bbc.com/news/uk-england-oxfordshire-41265057}{Bakhshālī} (Pakistan) in 1881, which has been carbon-dated to the 3\rd{} or 4\th{} century. The manuscript contains the earliest known version of the modern symbol for zero, a circular dot. It is conjectured that the decimal place system was inspired by the Chinese counting-board method, though convincing proof has yet to be uncovered. Regardless of attribution, Chinese mathematicians were copying the method by the 8\th{} century.\smallbreak
The examples below are better understood than the Bakhshālī manuscript and come from Gwalior (northern India) around \AD 876.
\begin{center}
	\begin{tabular}{ccccccccccc}
		0&1&2&3&4&5&6&7&8&9&10\\
		\IndiaGzero&\IndiaGone&\IndiaGtwo&\IndiaGthree&\IndiaGfour&\IndiaGfive&\IndiaGsix&\IndiaGseven&\IndiaGeight&\IndiaGnine&\IndiaGten
	\end{tabular}
\end{center}

The similarity with modern numerals is clear;  0, 1, 2, 3, 4, 7, 9, 10 are very familiar. Zero has evolved from the Bakhshālī dot to a hollow circle. The symbols for 2 and 3 are conjectured to have developed in an attempt to write earlier versions (e.g.{} the Brahmi numerals) cursively; try writing three horizontal strokes quickly\ldots
\smallbreak
The system is fully positional. Below are the numbers 270 and 30984:
\[
	\IndiaGtwo\IndiaGseven\IndiaGzero\qquad\qquad
	\IndiaGthree\IndiaGzero\IndiaGnine\IndiaGeight\IndiaGfour
\]
Sanskrit is written left-to-right, as are our modern numbers, with the leftmost digits representing the largest powers of 10. Note how zero is used as a placeholder to clarify position so that, e.g., 27, 207, and 270 are clearly distinguishable.
\goodbreak

\boldinline{Zero}
On the right is a table of modern Sanskrit names and numerals; the digits and names are certainly similar to their Gwalior counterparts.\par

\begin{minipage}[t]{0.59\linewidth}\vspace{-5pt}
	The Sanskrit \emph{shuunyá} means \emph{void} or \emph{emptiness.} It is related to \emph{svi} (hollow), which in turn derives from an ancient word meaning \emph{to grow.} This reflects a major idea within religions of the area, with the void being the source of all things, of creation and creativity. Contemplation of the void (the doctrine of Shunyata) is recommended before composing music, creating art, etc. This contrasts with the Abrahamic religions where the void is something to be feared; an early conception of hell was the eternal absence of God.
\end{minipage}
\hfill
\begin{minipage}[t]{0.4\linewidth}\vspace{-13pt}
	\flushright
	\includegraphics[scale=0.73]{sanskritnames}
\end{minipage}
\smallbreak
The Gwalior numerals travelled westwards, with Europe eventually inheriting the system via Islam; as such they are today known the \emph{Hindu--Arabic} numerals. Here is a short version of the etymological journey of zero into European languages.
\begin{itemize}
  \item \emph{Shunya} was transliterated to \emph{sifr} in Arabic where the double-meaning persisted: \emph{al-sifr} was the number zero, while \emph{safira} meant \emph{it was empty.}
  \item The term came to Europe in the 12\th{}-13\th centuries courtesy of Fibonacci where it became \emph{cifra.} This was blended with \emph{zephyrum} (\emph{west wind}/\emph{zephyr}) providing an alternate spelling.
  \item Cifra ultimately became the words \emph{cipher} (English), \emph{chiffre} (French) and \emph{ziffer} (German), meaning a figure, digit, or code.
  \item Zephyrum became \emph{zefiro} in Italian and \emph{zero} in Venetian.
\end{itemize}
Zero and the Hindu--Arabic numerals also travelled eastwards, with Qin Jiushao introducing the zero symbol into China in the 13\th{} century.
\bigbreak

Our modern understanding of zero is a fusion of several concepts:
\begin{description}
	\item[\normalfont\emph{Numerical positioning}] For instance, to distinguish 101 from 11.
	\item[\normalfont\emph{Absence of a quantity}] 101 contains no 10's.
	\item[\normalfont\emph{Symbol}] First a dot (\emph{bindu}), then a circle (\emph{chidra/randhra} meaning \emph{hole}). The relationship between \emph{shunya} and a symbol was established by \AD{2-300}, as this quote from \AD{400} (Vasavadatta) illustrates
	\begin{quote}
		The stars shone forth, like zero dots [shunya-bindu] scattered as if on a blue rug. The Creator reckoned the total with a bit of the moon for chalk.
	\end{quote}
	\item[\normalfont\emph{Mathematical operations}] By the time of Brahmagupta (7\th\,C.), mathematical texts often contained a section called \emph{shunya-gania,} with computations involving zero, including addition, multiplication, subtraction, effects on $\pm$-signs, division and the relationship with $\infty$ (\emph{ananta}). In the 12\th\,C., Bhaskaracharya stated:
	\begin{quote}
	If you were to divide by zero you would get a number that was ``as infinite as the god Vishnu.''
	\end{quote}
\end{description}


Other ancient cultures had one or more of these aspects of zero, but the Indians were the first to put them all together.
\begin{itemize}
  \item The Egyptian hieroglyph \emph{nfr} (beautiful/complete) indicated zero remainder in calculations as early as 1700\BC{} and was also used as a reference point/level in buildings.
  \item Very late in Babylonian times, a placeholder symbol was used to separate powers of 60. It was not used as a number.
  \item With the Chinese counting board, an empty space served as a placeholder.
  \item Various Mesoamerican cultures, such as the Maya, had a zero symbol that was used as a placeholder, particularly when writing dates. 
%   \item Greeks: obsessed with geomtery to exclusion of other things. Not interested in number systems, saw no use for zero. Arrogant quotes:
%   \begin{center}
%   Arithmetic should only be taught in democracies for it ``dealt with relations of equality''\\
%   Geometry the natural study of oligarchies, for ``it demonstrated the proportions with inequality''
%   \end{center}
\end{itemize}


\boldsubsubsection{`Real' Indian Mathematics}\phantomsection\label{pg:Mahavedi}

Indian mathematicians made great progress on several fronts, not merely the decimal place system.\par
\begin{minipage}[t]{0.55\linewidth}\vspace{0pt}
	Much ancient work was influenced by religion. The \emph{sulbasutras} were written during pre-Hindu times and contained instructions for laying out altars using ruler-and-compass constructions. These could be quite complex, as the construction of the base of the \emph{Mahavedi} (great altar) shows: The center line is divided left-to-right in the ratio
	\[
		1:7:12:11:5
	\]
	and the altar contains five distinct Pythagorean triples.
\end{minipage}
\hfill
\begin{minipage}[t]{0.42\linewidth}\vspace{0pt}
	\flushright
	\includegraphics[scale=0.9]{india-sulba}
\end{minipage}
\medbreak

Of particular importance to our narrative is Indian work on trigonometry. Here are some highlights:
\begin{itemize}
  \item The early 5\th\,C.{} text \emph{Paitāmahasiddhānta} is assumed to be an extension of Hipparchus' work, since it contains a table of chords based on a circle of radius 57,18; rather than Ptolemy's 60.
  \item Indian mathematicians instituted the use of \emph{half-chords,} in line with our modern understanding of sine. Indeed the word \emph{sine} is the result of a long sequence of (mis)translations and transliterations via Arabic and Latin from the Sanskrit \emph{jyā-ardha} (\emph{chord-half}).\footnote{Amazingly this became related to the word \emph{sinus} meaning \emph{bay, gulf} or \emph{bosom}!} The Indians also began to distinguish `base sine' and `perpendicular sine' (cosine).
  \item Created tables of sines/half-chords from \ang{0} to \ang{90} in steps of $\ang{3\frac 34}$, using linear interpolation to approximate values in between. By 650, Bhramagupta had much better approximations, using quadratic polynomials to interpolate. By 1530, Indian mathematicians had discovered cubic and higher approximations (essentially Taylor polynomials 130 years before Newton) for even greater accuracy of sine, cosine and arctangent.
\end{itemize}

Navigation was one of the drivers of this development. While Mediterranean sailors rarely strayed long out of sight of land, the Indians sailed the ocean and required accurate measurements to find their latitude.

\iffalse
\subsubsection*{Other Systems}

With same/similar symbols; other names. E.g. 0=shunya means ``void''\\
1 = candra (moon) or bhumi (earth)\\
2 = netra (eyes) or paska (wings of a bird), or other pairs\\
Aesthetically pleasing/literary way of counting but very hard to use: needed familiarity with particular writer's style\\

Aryabhata (b. 476) introduced alphabetical scheme, building (sometimes unpronouncable!) words where each letter/sound represented a number.\\
Refined/popularized by Varurici of the Kerala school of Astronomy (same time):\\
Consonants in order represent numbers 0--9, then repeat. Vowels have no numerical value unless unpreceeded by a consonant in which case reads 0. Use vowels and available consonants to help form meaningful phrases. I.e. (from Kunjunni Raja (1963m p.123), number 1,729,133 could be said either
\begin{center}
balakalatram saukhyam: ``the company of a young woman is sheer happiness''\\
lingavyadhir asahyah: ``the demise of sexual virility is unbearable''
\end{center}

Reflects oral traditions and strict links between literacy and numeracy.





\subsubsection*{Summary}

\begin{tabular}{l p{120pt} p{150pt} p{85pt}}
Period&Main Events&Mathematics&Notable Mathematicians\\\hline
3000--1500BC&Indus Valley Civilizaton&Weights, artistic designs, brick technology&\\
1500--500BC&Aryans arrive, Hindu civilisation, records begin&Astronomy, arithmetic, Vedic geometry&Baudhayana, Apastamba, Katyayana\\
500-200BC&Rise of Buddhism and Jainism, contact with Persia. Mauryan Empire culminating in Asoka who spread Buddhism abroad&Vedic Math declines with end of ritual sacrifice. Jainia math: number theory, permutations and combinations, binomial theorem, astronomy&\\
200BC--AD 400&India divided (North, South, Punjab). Buddhism main influence on
art/sculpture&Jainia math: rules of mathematical operations, decimal place
notation, zero, algebra (simple, simultaneous, quadratic equations), square
roots, negative signs&\\
400--1200&Imperial Guptas reach height (606--647). Flowering of Indian science, philosophy, medicine, logic, grammar, literature&Bhakshali Manuscript. Aryabhaitya, etc\ldots&Aryabhata I, Varahamihira, Bhaskara I, Bhramagupta, Sridhara, Mahavira, Bhaskara II (Bhaskaracharya)\\
1200--1600&Early Muslim dynasties, Sikhism&Decline of math/learning in North. Kerala school of astronomy + math, infinite series and analysis&Narayana, Nadhava, Nilakantha
\end{tabular}

\fi

\begin{exercises}{}{}
	\exstart The \emph{Mahavedi} (pg.\,\pageref{pg:Mahavedi}) contains five Pythagorean triples; find them.
	% \begin{gather*}
	% (5,12,13),\qquad (12,16,20),\qquad (12,35,37),\\
	% (15,20,25),\qquad (15,8,17)
	% \end{gather*} 
	\begin{enumerate}\setcounter{enumi}{1}
	  \item To simplify square root expressions, Bhaskara used the formula
	  \[
	  	\sqrt{a+\sqrt b}=\sqrt{\frac 12\Big(a+\sqrt{a^2-b}\Big)} +\sqrt{\frac 12\Big(a-\sqrt{a^2-b}\Big)}
	  \]
	  Prove Bhaskara's formula and use it to simplify $\sqrt{2+\sqrt 3}$.
	  
	  
		\item%[8-4]
	  Here is an Indian method for `finding' a circle whose area is equal to a given square.\par
	  \begin{minipage}[t]{0.62\linewidth}\vspace{-5pt}
		  In a square $ABCD$, let $M$ be the intersection of the diagonals. Draw a circle with $M$ as the center and $MA$ the radius; let $ME$ be the radius of the circle perpendicular to the side $AD$ and cutting $AD$ at $G$. Let $GN=\frac 13 GE$. Then $MN$ is the radius of the desired circle.\par
		  Show that if $AB=s$ and $MN=r$, then
		  \[
		  	\frac rs=\frac{2+\sqrt 2}6
		  \]
		  Show that this implies a value for $\pi$ equal to $3.088311755$.
	 	\end{minipage}
	 	\hfill
	 	\begin{minipage}[t]{0.35\linewidth}\vspace{-5pt}
	  	\flushright
	  	\includegraphics[scale=0.9]{india-squarecirc}
	  \end{minipage}  
	
	  
	  \item%[8-12*]
	  Solve the following problem of Mahāvīra.
	  \begin{quote}
	  	Of a collection of mango fruits, the king took 1/6; the queen took 1/5 of the remainder, and the three chief princes took 1/4, 1/3, 1/2 of what remained at each step. The youngest child took the remaining three mangoes. O you, who are clever in working miscellaneous problems on fractions, give out the measure of that collection of mangoes.
	  \end{quote}
	\end{enumerate}
\end{exercises}


\clearpage

\subsection{Islamic Mathematics and Algebra}\label{sec:islamalgebra}

Muhammad ibn Abdullah was born in Mecca (modern Saudi Arabia) in 570. Around 610 he began preaching \emph{Islam} (\emph{submission to the will of God}), the third major Abrahamic religions (following Judaism and Christianity). After years of persecution and exile, he returned with an army, conquering Mecca a few years before his death in 632.\smallbreak

Muhammad's successors expanded the caliphate (empire) through military conquest. The speed of the Muslim conquests is illustrated below. At the time of his death, the \textcolor{Maroon}{Arabian peninsula} was Islamic. By 660 Islam had reached \textcolor{OrangeRed}{Libya and most of Persia,} and by 750 extended from \textcolor{Goldenrod}{Iberia \& Morocco to Afganistan \& Pakistan.} In later times there were serious schisms\footnote{In particular between the Sunni and Shia branches of the faith. Much of the modern-day tension between Saudi Arabia and Iran stems from this rupture.} and several successor empires emerged, the longest-lasting of which was the Ottoman Empire (c.\,1300--1922). Even though centralized political control ended long ago, Islam remains dominant in the region shown (with the notable exceptions of Spain and Portugal) and over a greater region of Africa and south-east Asia.
\begin{center}
	\includegraphics[width=0.7\linewidth]{caliphate.png}
\end{center}

As with the Romans, early Muslims permitted conquered peoples to maintain their culture, especially if they were Jews or Christians (\emph{people of the book}), provided they acknowledged their overlords and paid taxes. Those who converted to Islam were welcomed as full citizens. Many of the great Islamic thinkers were just such people, born on the periphery and travelling to the great centers of learning, particularly Baghdad during the Islamic golden age (8\th--13\th centuries). Knowledge was also absorbed from Alexandria and western India (Pakistan). In the mid-700s paper-making came from China, greatly facilitating the dissemination and consolidation of knowledge. Schools (\emph{madrassas}) reflected a strong cultural focus on learning.\smallbreak

The Islamic golden age overlapped the European \emph{dark ages} (c.\,500--1200) following the fall of Rome, during which European philosophical development stagnated. By 1200, the crusades\footnote{A series of religious--military campaigns 1096--1291 with the goal of wresting control of the Holy Land, particularly Jerusalem, from Islam.} were well underway and Islam had come to be seen as an enemy of Christian Europe. Ironically, the infusion of knowledge that came to Europe from Islam around this time helped spur the European renaissance \& later scientific revolution. Among European scholars almost to the present day, it was fashionable to credit Islam merely with the \emph{preservation} of ancient `European' knowledge; a claim both fanciful and chauvinistic, but plainly stemming from medieval animosity.

\goodbreak


\boldsubsubsection{Algebra \& Algorithms}

Concepts of proof and axiomatics were learned from Greek texts such as the \emph{Elements.} Like the Greeks, Islamic scholars gave primacy to geometry and proved algebraic relations in a geometric manner.\footnote{Like Book II of the \emph{Elements.} 
Such Greek texts were venerated by Islamic scholars; recognizing the depth of Ptolemy's work on astronomy and trigonometry, they bestowed the name by which it is now known, the \emph{Almagest} (\emph{Great Work}).} Practical and accurate calculation was more important than it was to the Greeks, and great advances were made in this area. This included completing the development of the Indian decimal place system (hence the dual credit \emph{Hindu--Arabic} numerals).\smallbreak
The second most obvious legacy of Islamic mathematics is encountered daily in every mathematics classroom. \emph{Algebra}\footnote{Many words beginning \emph{al-} are of Arabic origin (alkali, albatross, etc.), as are others that have been latinized (elixir).} comes from the Arabic \emph{al-ğabr}, meaning \emph{restoring}. It originally referred to moving a deficient (negative) quantity from one side of an equation to another. A second term \emph{al-muqabala} (\emph{comparing}/\emph{balancing}) meant to subtract the same positive quantity from both sides of an equation.
\begin{quote}
	\begin{tabular}{l@{\quad}l}
		\emph{Al-ğabr}:&$x^2+7x=4-2x^2\implies 5x^2+7x=4$\\[4pt]
		\emph{Al-muqabala}:&$x^2+7x=4+5x\implies x^2+2x=4$
	\end{tabular}
\end{quote}
Islamic scholars did not use \emph{symbols} or \emph{equations} in a modern sense; statements were instead written out in sentences.


\boldinline{Muhammad ibn Mūsā al-Khwārizmī (780--850)}
	
Born near the Aral Sea in modern Uzbekistan, al-Khwārizmī eventually became chief librarian at the great school of learning, the \emph{House of Wisdom,} in Baghdad. His \emph{Compendious book on the calculation by restoring and balancing}\footnote{\emph{Al-kitāb al-mukhtasar fī hisāb al-ğabr wa’l-muqābala.}} (820) is a synthesis of Babylonian methods with the Euclidean axiomatic approach; an algorithm demonstrated a solution, followed by a geometric proof. After being translated into Latin in the 1100s it became a standard textbook of European mathematics, displacing Euclid in places due to its greater emphasis on practical calculation. The word \emph{algorithm} reflects its importance: the Latin \emph{dixit algorismi} literally means \emph{al-Kwārizmī says.} Here is his approach to the quadratic equation $x^2+4x=60$, or, more properly:
\begin{quote}
	What must be the square which, when increased by four of its roots, amounts to sixty?
\end{quote}


\begin{minipage}[t]{0.7\linewidth}\vspace{-4pt}
	The algorithm may be applied to \emph{any} equation of the form $x^2+ax=b$ where $a,b>0$: here $a$ is the number of `roots,' and $b$ the total `amount.'
	\begin{itemize}\itemsep0pt
	  \item Halve the number of roots\hfill \big($2=\frac 12a$\big)
	  \item Multiply by itself \hfill \big($4=\frac 14a^2$\big)
	  \item Add to the total amount \hfill \big($64=\frac 14a^2+b$\big)
	  \item Take the root of this \hfill \Big($8=\sqrt{\frac 14a^2+b}$\Big)
	  \item Subtract half the number of roots \hfill \smash[t]{\Big($6=\sqrt{\frac 14a^2+b}-\frac a2$\Big)}
	\end{itemize}
\end{minipage}
\hfill
\begin{minipage}[t]{0.29\linewidth}\vspace{-3pt}
	\flushright
	\includegraphics[scale=1]{islam-quad}
\end{minipage}
\smallbreak
Al-Kwārizmī essentially constructs the quadratic formula $=\frac{-a+\sqrt{a^2+4b}}2$, while the pictorial justification is Euclid's (\emph{Elements}, Thm II.\,4). Hopefully the geometry is obvious: the square has been increased by four of its roots and the algorithm is simply `completing the square.'
\goodbreak

%Other algorithms were given in order to solve every type of quadratic.\smallbreak


Other mathematicians went far further. For instance, Abū Kāmil (Egypt 850--930) generalized Euclid's Book II geometric-algebra arguments to permit substitution, provided the resulting equation was quadratic; for instance
\begin{gather*}
	\text{If }y=\frac{1+x}{3+x}\ \text{ and }\ y^2+y=1\implies y=\frac{\sqrt 5 -1}2\text{, then}\\[2pt]
	\left(\frac{1+x}{3+x}\right)^2+\frac{1+x}{3+x}=1\implies x=\sqrt 5
\end{gather*}
In practice, he combined several algorithms of al-Khwārizmī, each of which was based on geometry, but when combined could no-longer be straightforwardly be justified geometrically. This method of substitution was an early step towards establishing the modern primacy of algebra and number over geometry and length.\smallbreak
The most famous Islamic mathematician of the next several centuries was almost certainly Omar Khayyam (1048--1131), who produced ground-breaking work on the solution of cubic equations, astronomy, the binomial theorem, and irrational numbers. %He did early work which laid the foundations for non-Euclidean geometry and the theory of irrational numbers.


\begin{exercises}{}{}
	\exstart %[9-4*]
	Solve the equations $\frac 12x^2+5x=28$ and $2x^2+10x=48$ using al-Khwārizmī's methods (first multiply or divide by 2).
	  
	\begin{enumerate}\setcounter{enumi}{1}
	  \item%[9-2*]
	  \label{exs:alkalgebra}
	  Al-Khwārizmī gives the following algorithm for solving the equation $bx+c=x^2$.\\
	  \begin{minipage}[t]{0.6\linewidth}\vspace{0pt}
	  \begin{itemize}\itemsep0pt
	    \item Halve the number of roots.
	    \item Multiply this by itself.
	    \item Add this square to the number.
	    \item Extract the square root.
	    \item Add this to half the roots.
	  \end{itemize}
	  Translate this into a formula. Give a geometric argument for the validity of the approach using the picture: $HC$ has length $b$ where $G$ is the midpoint, rectangle $ABRH$ has area $c$, $KHGT$ and $AMLG$ are squares, while the large square $ABDC$ has side length $x$.
	  \end{minipage}
	  \hfill
	  \begin{minipage}[t]{0.39\linewidth}\vspace{0pt}
	  	\flushright
	  	\includegraphics{hw-alk}
	  \end{minipage}
	  
	  
	  \item%[9-6]
	  Solve the following problems by Abū Kāmil (use modern algebra!).
	  \begin{itemize}
	    \item[(a)] Suppose 10 is divided into two parts and the product of one part by itself equals the product of the other part by the square root of 10. Find the parts.
	    \item[(b)] Suppose 10 is divided into two parts, each of which is divided by the other, and the sum of the quotients equals the square-root of 5. Find the parts.
	  \end{itemize}
	\end{enumerate}
\end{exercises}

\clearpage



\subsection{Spherical Trigonometry and the \emph{Qibla}}

Late 8\th{} century Indian work on trigonometry, linking back to Hipparchus, was known in Baghdad, as was the work of Ptolemy. Islamic scholars were interested in trigonometry for reasons beyond astronomy. A primary requirement in Islam is to face the Ka'aba in the Great Mosque at Mecca when at prayer: this is the \emph{qibla} (\emph{direction} in Arabic). A mosque is typically built so that one wall faces Mecca for convenience. In Muhammad's time (when Muslims faced Jerusalem not Mecca), determining the \emph{qibla} was relatively easy, though as Islam spread the curvature of the earth made determination more difficult. The resolution of this problem motivated Islamic mathematics for centuries.



\boldinline{Terminology and Trigonometric Tables}

Scholars worked with the Indian \emph{half-chord} (sine), and with circles of various radii. Al-Battānī (c.\,858--929) introduced an early version of \emph{cosine} as the \emph{complementary half-chord} for angles less than \ang{90}. He also worked with a version of the modern function \emph{versine}:\footnote{\emph{Versed sine} refers to the measurement of a length in a re\emph{versed} direction (perpendicular) to that of sine.}
\[
	\operatorname{versin}\theta=1-\cos\theta
\]
Al-Bīrūnī (973--1048) defined versions of tangent, cotangent, secant and cosecant by projecting from a gnomon (sundial) onto either a horizontal or vertical plane. In the second picture below, the gnomon is the vertical stick of length 1. With this definition, al-Bīrūnī moves towards the modern consideration of trigonometry in terms of \emph{triangles} rather than circles.
\begin{center}
  \includegraphics{islam-versin}\qquad\qquad
  \includegraphics{islam-csc}
\end{center}\phantomsection\label{pg:sineconcave}

Trigonometric tables with improved accuracy over Ptolemy were created for all these `functions.' By applying the half-angle formula many times and using the fact that
\[
	\sin(\alpha+\beta)-\sin\alpha<\sin\alpha-\sin(\alpha-\beta)
\]
whenever\footnote{This is simply the downwards concavity of the sine function.} $\ang{0}<\alpha-\beta<\alpha+\beta<\ang{90}$, Abū al-Wafā (940--998) and his descendants were able to compute sine \& tangent values for every minute of arc accurate to five sexagesimal places!
\goodbreak


\boldinline{Calculating the \emph{Qibla}}

In what follows we observe several conventions:
\begin{itemize}\itemsep0pt
  \item A single letter $A$ refers to a \emph{point} or to the \emph{angle measure} in a triangle with vertex $A$.
  \item $AB$ means the \emph{segment} of the great-circle joining points $A,B$ or its \emph{arc-length.} A \emph{spherical triangle} $\triangle ABC$ comprises three points on a sphere joined by segments of great-circles.
  \item $\cl{AB}$ means the \emph{straight line} joining $A,B$ with \emph{length} $\nm{AB}$.
  \item All results are modernized and applied to a unit sphere. The arc-length along a great-circle therefore equals the central angle subtended by that arc in radians: $AB=\measuredangle AOB$. To help visualize things, 3D movable versions of all pictures can be found by clicking on them\ldots 
\end{itemize}

\begin{minipage}[t]{0.75\linewidth}\vspace{-5pt}
	Ptolemy and the Indians had already done some relevant work, though Ptolemy's approach relies heavily on Menelaus' Theorem (c.\,100\AD).
	
	\begin{thm*}{Menelaus}{}
	For the pictured configuration of spherical triangles on a sphere of radius 1,
	\[
		\frac{\sin CE}{\sin AE}=\frac{\sin CF}{\sin DF}\cdot\frac{\sin BD}{\sin AB}
	\]
	\end{thm*}
	
	Computations are difficult with Menelaus' Theorem, since one needs to create several new spherical triangles. Al-Wafā simplified things considerably with an alternative result.
	\end{minipage}
	\hfill
	\begin{minipage}[t]{0.24\linewidth}\vspace{-5pt}
		\flushright
		\href{http://math.uci.edu/~ndonalds/math184/islam-menelaus.html}{\includegraphics{islam-menelaus}}
\end{minipage}
\medbreak




\begin{thm*}{Al-Wafā}{}
	If $\triangle ABC$ and $\triangle ADE$ are spherical triangles with right angles at $B,D$ and a common acute angle at $A$, then
	\[
		\frac{\sin BC}{\sin AC}=\frac{\sin DE}{\sin AE}
	\]
\end{thm*}

In fact these ratios equal $\sin\alpha$ where $\alpha$ is the acute angle, though al-Wafā didn't say this.

\begin{proof}
	Let $O$ be the center of the sphere. Project $C$ orthogonally to the plane containing $O,A,B$ to produce $K$, then project $K$ to $\cl{OA}$ to get $L$.\par
	\begin{minipage}[t]{0.55\linewidth}\vspace{-5pt}
		Consider the right-angled \textcolor{red}{planar triangle $CKL$.} Since $\alpha$ is the angle between two planes, we have $\textcolor{red}{\alpha=\measuredangle CLK}$. Moreover
		\begin{gather*}
		\nm{CK}=\sin\measuredangle COK=\sin\measuredangle COB=\sin BC\\
		\nm{CL}=\sin\measuredangle COL=\sin\measuredangle COA=\sin AC
		\end{gather*}
		The usual sine formula says
		\[
			\sin\textcolor{red}{\alpha}=\frac{\nm{CK}}{\nm{CL}}=\frac{\sin BC}{\sin AC}
		\]
	\end{minipage}
	\hfill
	\begin{minipage}[t]{0.4\linewidth}\vspace{-15pt}
		\flushright
		\href{http://math.uci.edu/~ndonalds/math184/islam-alwafa.html}{\includegraphics{islam-alwafa}}
	\end{minipage}
	\par\vspace{-3pt}
	The same ratio is obtained for $\triangle ADE$.
\end{proof}

\goodbreak

Solving triangles essentially means the sine and cosine rules, both of which follow by dropping perpendiculars. Al-Wafā's result quickly recovers the spherical sine rule.

\begin{minipage}[t]{0.65\linewidth}\vspace{0pt}
	\begin{cor*}{Sine rule}{}
	If $a,b,c$ are the side-lengths of a spherical triangle with angles $A,B,C$, then
	\[\frac{\sin a}{\sin A}=\frac{\sin b}{\sin B}=\frac{\sin c}{\sin C}\]
	\end{cor*}
	
	\begin{proof}
	Drop a perpendicular to $H$ from $C$. Al-Wafā says
	\[\sin B=\frac{\sin h}{\sin a}\quad\text{and}\quad \sin A=\frac{\sin h}{\sin b}\]
	Eliminate $\sin h$ for the first equality. The rest is symmetry.
	\end{proof}
\end{minipage}
\hfill
\begin{minipage}[t]{0.34\linewidth}\vspace{0pt}
	\flushright
	\href{http://math.uci.edu/~ndonalds/math184/islam-sine.html}{\includegraphics{islam-sine}}
\end{minipage}
\medbreak

\begin{minipage}[t]{0.65\linewidth}\vspace{0pt}
	Al-Wafā's proof was similar, though a little more complicated. He extended $AB$ and $BC$ to quarter circles resulting in a spherical triangle with right angles at $D$ and $E$. Since $DE$ is an arc with central angle $B$, we have $DE=\sin B$. Al-Wafā's theorem implies
	\[
		\frac{\sin h}{\sin a}=\frac{\sin B}{\sin \ang{90}}\implies \sin h=\sin a\sin B
	\]
	Mirroring this by extending $AB$ past $B$ and equating the $\sin h$ terms yields the result.
\end{minipage}
\hfill
\begin{minipage}[t]{0.34\linewidth}\vspace{-15pt}
	\flushright
	\href{http://math.uci.edu/~ndonalds/math184/islam-sine2.html}{\includegraphics{islam-sine2}}
\end{minipage}
\smallbreak

\begin{minipage}[t]{0.65\linewidth}\vspace{0pt}
	Armed with these results, al-Wafā could solve spherical triangles, though his method was rather complicated and required several auxiliary triangles. Al-Bīrūnī simplified matters by invoking the cosine rule. We apply his method to find the \emph{qibla} from a location $L$ on the Earth's surface which, for simplicity, we assume to have radius 1.\smallbreak
	Let $M$ be Mecca and $N$ the north pole. We wish to compute $\beta$ (the initial bearing from $L$ to $M$). Our initial data are the latitudes and longitudes of $L,M$, specifically:
	\begin{itemize}\itemsep0pt
	  \item $\alpha$ is the difference in the longitudes.
	  \item $b,c$ are the \emph{colatitudes}\footnotemark{} of $M,L$ respectively. 
	\end{itemize} 
\end{minipage}
\hfill
\begin{minipage}[t]{0.34\linewidth}\vspace{0pt}
	\flushright
	\href{http://math.uci.edu/~ndonalds/math184/islam-qibla.html}{\includegraphics{islam-qibla}}
\end{minipage}
\medbreak

\footnotetext{\emph{Colatitude} is measured southwards from the north pole, adn thus equals $\ang{90}-$latitude. Since the sphere has radius 1, the arc-lengths $b,c$ equal the colatitudes in radians.}

The cosine rule follows from Ptolemy's Theorem (pg.\,\pageref{pg:ptolemythm}). Extend $NL$ to $Q$ with the same latitude as $M$. Similarly let $P\in NM$ have the same latitude as $L$. By symmetry, $L,P,Q,M$ are \emph{coplanar,} whence the quadrilateral $LPQM$ lies on the intersection of a plane and a sphere: a circle! Measured as straight lines (chords) and using symmetry ($\nm{PQ}=\nm{LM}$ and $\nm{LQ}=\nm{PM}$), Ptolemy says
\[
	\nm{LM}\nm{PQ}=\nm{LQ}\nm{PM}+\nm{LP}\nm{QM} \implies \nm{LM}^2=\nm{LQ}^2+\nm{LP}\nm{QM}
\]
\goodbreak

The great circle arc-lengths on the sphere are related to straight distances via the usual chord relations; e.g.,\par
\begin{minipage}[t]{0.65\linewidth}\vspace{-5pt}
	\[
		\nm{LM}=\crd LM=2\sin\frac{\textcolor{red}{LM}}2
	\]
	Thus Ptolemy's theorem becomes a relation between \emph{arc-lengths}
	\[
		\sin^2\frac{\textcolor{red}{LM}}2=\sin^2\frac{b-c}2+\sin\frac{LP}2\sin\frac{QM}2
	\]
	By bisecting $\alpha$ we obtain two pairs of right-triangles; al-Wafā tells us that
	\[
		\sin\frac{\textcolor{red}{\alpha}}2=\frac{\sin\frac{LP}2}{\sin \textcolor{blue}{c}}=\frac{\sin\frac{QM}2}{\sin \textcolor{Green}{b}}
	\]
\end{minipage}
\hfill
\begin{minipage}[t]{0.34\linewidth}\vspace{0pt}
	\flushright
	\href{http://math.uci.edu/~ndonalds/math184/islam-cosine.html}{\includegraphics{islam-cosine}}
\end{minipage}
\bigbreak

\begin{minipage}[t]{0.7\linewidth}\vspace{0pt}
	whence
	\[
		\sin^2\frac{\textcolor{red}{LM}}2=\sin^2\frac{b-c}2+\sin^2\frac{\textcolor{red}{\alpha}}2\sin \textcolor{blue}{c}\sin \textcolor{Green}{b}\tag*{($\ast$)}
	\]
	For final simplifications, apply the multiple-angle formulæ ($\sin^2x=\frac 12(1-\cos 2x)$ and $\cos(b-c)=\cos b\cos c+\sin b\sin c$).

	\begin{cor*}{Cosine rule}{}
	In a spherical triangle with sides $a,b,c$ and angle $\alpha$ opposite $a$, we have
	\[
		\cos a=\cos b\cos c+\sin b\sin c\cos\alpha
	\]
	\end{cor*}

\end{minipage}
\hfill
\begin{minipage}[t]{0.29\linewidth}\vspace{-10pt}
	\flushright
	\href{http://math.uci.edu/~ndonalds/math184/islam-cosine2.html}{\includegraphics{islam-cosine2}}
\end{minipage}
\bigbreak

Of course $\textcolor{red}{a=LM}$ in our triangle of interest. Given $L,M$, one uses the cosine rule to compute $a$ and then the sine rule
\[
	\frac{\sin \textcolor{Green}{b}}{\sin\textcolor{Green}{\beta}}=\frac{\sin \textcolor{red}{a}}{\sin\textcolor{red}{\alpha}}
\]
to compute the \emph{qibla} $\beta$. Whew!\medbreak

For fun, here is some real-world data: the co-ordinates of Mecca and London are \ang{21}25'\,N \ang{39}49'\,E and \ang{51}30'\,N 8'\,W respectively. This corresponds to
\[
	\textcolor{red}{\alpha=\ang{39}57'},\qquad \textcolor{Green}{b=\ang{68}35'},\qquad \textcolor{blue}{c=\ang{38}30'}
\]
We therefore have
\[
	\cos\textcolor{red}{a} =\cos \textcolor{Green}{\ang{68}35'}\cos \textcolor{blue}{\ang{38}30'}+\sin \textcolor{Green}{\ang{68}35'}\sin\textcolor{blue}{\ang{38}30'}\cos \textcolor{red}{\ang{39}57'} \implies \textcolor{red}{a=\ang{43.110}}
\]
Since Earth's circumference is 24,900 miles, the London $\to$ Mecca distance is $\frac{43.110\times 24,900}{360}=2,981$ miles. Finally we find the \emph{qibla}
\[
	\textcolor{Green}{\beta}=\ang{180} -\sin^{-1}\frac{\sin\textcolor{red}{\alpha}\sin\textcolor{Green}{b}}{\sin \textcolor{red}{a}}=\textcolor{Green}{\ang{118}59'}
\]
where we subtracted from \ang{180} since London is north of Mecca. Check it yourself at the \href{http://www.gcmap.com/mapui?P=LON-QCA}{Great Circle Mapper} (the website uses airports so the result uses slightly different initial data).

\goodbreak

\boldsubsubsection{Spherical Trigonometry: Cheat Sheet!}

Let $\triangle ABC$ be a spherical triange with side-lengths $a,b,c$ on a sphere of radius 1.\smallbreak

\emph{Basic trigonometry}: if $\triangle ABC$ is right-angled at $C$ (Al-Wafā essentially proved the first of these),
\[
	\sin A=\frac{\sin a}{\sin c}\qquad \cos A=\frac{\tan b}{\tan c}\qquad \tan A=\frac{\tan a}{\sin b}
\]

\emph{Sine rule} (Al-Wafā)
\[
	\frac{\sin A}{\sin a}=\frac{\sin B}{\sin b}=\frac{\sin C}{\sin c}
\]

\emph{Cosine rule} (Al-Bīrūnī)
\[
	\cos c=\cos a\cos b+\sin a\sin b\cos C
\]
The spherical Pythagorean Theorem is $\cos c=\cos a\cos b$ \ ($C=\ang{90}$).\smallbreak

Combining these results allows one to `solve' spherical triangles. If the sphere has radius $r$, simply divide all lengths by $r$ before applying the results; e.g.,
\[
	\sin A=\frac{\sin (a/r)}{\sin(c/r)}
\]
Note that as $r\to\infty$, we have $\sin(a/r)\approx \frac ar$ and $\cos(a/r)\approx 1-\frac{a^2}{2r^2}$, which recovers the `flat' versions of the above statements.


\boldinline{Examples}
\begin{enumerate}
	\item On a sphere of radius 1, an equilateral triangle has side length $\frac\pi 3$. Splitting it in half creates two right-triangles with adjacent $\frac\pi 6$ and hypotenuse $\frac\pi 3$. The angles in the triangle are therefore
	\[
		\alpha=\cos^{-1}\frac{\tan\frac\pi 6}{\tan\frac\pi 3}=\cos^{-1}\frac 13\approx \ang{70.53}
	\]
	The angle sum in the triangle is three times this: \ang{211.59}.
	
	\item On Earth's surface, an airfield is at position $C$ and two planes are at $A$ and $B$. The bearings and distances to the aircraft are \ang{45}, 2000 miles, and \ang{90}, 4000 miles respectively. Find the distance between the aircraft.\smallbreak
	This is just the cosine rule! We have a triangle with sides 2000 and 4000 and angle \ang{45} between them. If $r=4000$ miles is the radius of the Earth, then
	\begin{gather*}
		\cos\frac c{4000}=\cos\frac 1{2}\cos 1+\sin\frac 12\sin 1\cos\ang{45}\\
		\implies c=2833\text{ miles}
	\end{gather*}
	Note that this is a little closer (as expected) than the value (2947 miles) one would obtain from assuming a flat Earth!\smallbreak
	Modern navigators use a slightly different approach to measuring the distance between airplanes, since the error in estimating cosine of small values tends to make for a high level of inaccuracy (look up the \emph{haversine formula} if you're interested).
\end{enumerate}



\begin{exercises}{}{}
	\exstart A right-isosceles triangle on the surface of a unit sphere has equal legs of length $\frac\pi 4$. Find the length of the hypotenuse and the sum of the angles in the triangle. 
	\begin{enumerate}\setcounter{enumi}{1}
	  \item Explain the observation on page \pageref{pg:sineconcave} that
		\[
			\ang 0<\alpha-\beta<\alpha+\beta<\ang{90}\implies\sin(\alpha+\beta)-\sin\alpha<\sin\alpha-\sin(\alpha-\beta)
		\]
		is the downwards concavity of the sine function.
	
	  \item%[9-34*]
		Determine the \emph{qibla} for Rome (latitude \ang{41}53'\,N, longitude \ang{12}30'\,E).
	  
	  \begin{minipage}[t]{0.7\linewidth}\vspace{0pt}
		\item%[9-36*]
		Al-Bīrūnī devised a method for determining the radius $r$ of the earth by sighting the horizon from the top of a mountain of known height $h$. He would measure $\alpha$, the angle of depression from the horizontal to which one sights the apparent horizon. Show that
	  \[
	  	r=\frac{h\cos\alpha}{1-\cos\alpha}
	  \]
	  In a particular case, al-Bīrūnī measures $\alpha=34'$ from a mountain of height $652;3,18$ cubits. Assuming that a cubit equals $18''$, convert your answer to miles and compare with the modern value. Discuss the efficacy of this method.
	  \end{minipage}
	  \hfill
	  \begin{minipage}[t]{0.29\linewidth}\vspace{0pt}
	  	\flushright
	  	\includegraphics{hw-albiruni}
	  \end{minipage}
	  
	\end{enumerate}
\end{exercises}